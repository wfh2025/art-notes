\documentclass{../main.tex}{subfiles}
\begin{document}
%%%%%%%%%%%%%%%%%%%%%%%%%%%%%%%%%%%%%%%%%%%%%%%%%%%%%%%%%%%%%%%%%%%%%%%%%%%%%%%%%%%%%%%%%%%%%%%%%%%
\chapter{C++}

\section{面向对象}
\subsection{错误构造对象}
\inputCppFile[{错误构造对象方式}]{res/pl/cpp/code/001-constructor.cpp}

\begin{enumerate}[labelindent=0pt, itemsep=0pt, parsep=0pt, topsep=0pt, partopsep=0pt]
    \item ``A a()''这种\emph{无法构造对象}
    \item 编译器当成函数声明
\end{enumerate}

\subsection{正确调用无参构造函数}
\inputCppFile[{正确调用无参构造函数}]{ res/pl/cpp/code/002-constructor.cpp}

\subsection{无拷贝构造函数使用=赋值对象}
\inputCppFile[{无拷贝构造函数使用=赋值对象}]{ res/pl/cpp/code/003-constructor.cpp}
\begin{enumerate}[labelindent=0pt, itemsep=0pt, parsep=0pt, topsep=0pt, partopsep=0pt]
    \item 这里多产了一次析构函数,说明多产生一个对象
    \item 此处是直接赋值,若是堆内存指针,可能会异常
\end{enumerate}

\subsection{使用=调用拷贝构造函数}
\inputCppFile[{使用=调用拷贝构造函数}]{res/pl/cpp/code/004-constructor-copy.cpp}

\subsection{使用圆括号调用拷贝构造函数}
\inputCppFile[{使用圆括号调用拷贝构造函数}]{res/pl/cpp/code/005-constructor-copy.cpp}

\subsection{使用函数入参调用拷贝构造函数}
\inputCppFile[{使用函数入参调用拷贝构造函数}]{res/pl/cpp/code/006-constructor-copy.cpp}

\subsection{函数返回对象}
\inputCppFile[{函数返回对象}]{res/pl/cpp/code/007-constructor-copy.cpp}

\subsection{运算符等号重载}
\inputCppFile[{运算符等号重载}]{res/pl/cpp/code/008-operator-equal.cpp}
\begin{enumerate}[labelindent=0pt, itemsep=0pt, parsep=0pt, topsep=0pt, partopsep=0pt]
    \item 当前``=''没有产生新的对象
    \item 注意不要和前面的使用``=''拷贝构造函数混淆
\end{enumerate}


\section{智能指针}
\subsection{unique\_ptr基本使用}
\inputCppFile[{unique\_ptr基本使用}]{res/pl/cpp/code/009-unique-ptr.cpp}

\section{引用}
\subsection{引用不能重定位}
\inputCppFile[{引用不能重定位}]{res/pl/cpp/code/010-ref-not-redirect.cpp}

\section{初始化}
\subsection{初始化变量}
% \lstinputlisting[style = CodeLstStyleCpp, caption = {初始化变量}]{
%     res/pl/cpp/code/011-init.cpp
% }

\section{继承}
\subsection{简单继承}
% \lstinputlisting[style = CodeLstStyleCpp, caption = {简单继承}]{
%     res/pl/cpp/code/012-inherit.cpp
% }

\subsection{public继承}
% \lstinputlisting[style = CodeLstStyleCpp, caption = {public继承}]{
%     res/pl/cpp/code/013-inherit-public.cpp
% }



%%%%%%%%%%%%%%%%%%%%%%%%%%%%%%%%%%%%%%%%%%%%%%%%%%%%%%%%%%%%%%%%%%%%%%%%%%%%%%%%%%%%%%%%%%%%%%%%%%% 
\end{document}