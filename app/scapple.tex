\documentclass{../main.tex}{subfiles}
\begin{document}
%%%%%%%%%%%%%%%%%%%%%%%%%%%%%%%%%%%%%%%%%%%%%%%%%%%%%%%%%%%%%%%%%%%%%%%%%%%%%%%%%%%%%%%%%%%%%%%%%%%
\chapter{Scapple}
\section{概述}
\emph{Scapple}允许在虚拟纸张任意位置记录笔记,并能快速、轻松在这些笔记之间建立关联。
既能记录想法,又能梳理想法之间逻辑关系工具。

\section{基础}
\begin{lstlisting}[style=lstCodeStylePlainText]
- 要创建新的Scapple文档,请从"文件"菜单中选择"新建"(Cmd+N)。
- 在文档任意位置双击可创建新笔记(Cmd+Enter);
- 双击便签可以进行编辑,按Escape键结束编辑;
- 从Finder中拖入图像文件以将其添加到您的文档中。(文本文件也可以拖到文档中以导入文本);
点击便签可选中它,或在便签外点击并拖动以创建一个"marquee"矩形,拖动该该矩形可选中多个便签,按住Shift和Cmd点击也能选中多个便签;
- 拖动便签可移动其位置(若要移动多个便签,请先选中它们再拖动);
- 要删除便签,选中它们并按Delete键(从"编辑"菜单中选择"删除");
- 将一个笔记拖到另一个笔记上,即可在它们之间建立连接,重复操作可移除连接;
- 将一个便签拖放到另一个便签上时,按住Option可创建或移除箭头连接,拖放时按住Option+Cmd可使箭头指向相反方向,按住\texttt{Shift+Cmd}箭头双向指向;
- 双击创建新笔记使用相同修饰键,会使新笔记与任何选中的笔记建立连接(双击时按住Cmd键盘,可创建无箭头连接的新笔记);
抓住并拖动笔记之间的连接线,即可移动相互连接的笔记;
- 双击两个笔记之间的连接线,可创建一个与两者都相连的新笔记,并替换原来的连接;
- 点击并拖动便签的左边界或右边界可调整其大小。拖动按住Option可向两个方向调整大小(图片和背景形状的顶部和底部边界均可以拖动,背景形状的边角也可以拖动);
\end{lstlisting}

\section{格式与排列}
\begin{lstlisting}[style=lstCodeStylePlainText]
- 使用检查器可以更改笔记以及整个文档的外观。
- 通过"视图">"显示检查器"来显示检查器.(检查器中许多选项也可以在"格式"菜单中找到);
- 可以通过"格式"菜单的"笔记样式"子菜单保存和应用格式预设。
- 可以通过"格式"菜单的"笔记样式"子菜单保存和应用格式预设。可以通过一个命令应用多种格式选项,笔记样式也可以通过上下文菜单应用。
- Notes菜单包含辅助排列note的命令,例如对齐、统一宽度/高度、置于顶层/底层
- Tab键缩进选中的note;Shift-Tab减少缩进
\end{lstlisting}

\section{堆叠}
\begin{lstlisting}[style=lstCodeStylePlainText]
- 选中一条笔记后,Cmd+Return可创建一条新笔记,直接堆叠其下方,当笔记处于堆叠状态,若上方笔记被编辑或移出堆叠,垂直位置会自动调整以保证堆叠状态
- 可以通过"Notes"菜单选择堆叠来堆叠选中的笔记(笔记会堆叠在最先选中的笔记下方)。
- 要将现有笔记添加到一个堆叠中,首先在你希望放置该笔记的堆叠下方选择一个笔记,然后选中要移动的笔记;从“Notes"菜单或上下文菜单中选择"堆叠".
\end{lstlisting}

\section{背景形状}
\begin{lstlisting}[style=lstCodeStylePlainText, title={背景形状}]
- 背景形状用于包围notes(或其他背景形状),作为视觉辅助。
- 要创建背景形状,按住Ctrl单击,新建>背景形状。(如果按住Ctrl单击现有notes或形状,该命令变为"围绕所选内容新建背景形状");
- 通过检查器或Notes菜单为背景形状开启“磁性”选项后,任何与该形状重叠的notes都会被吸附到它上面,因此移动该形状时嘛,重叠的note也会随之移动.
\end{lstlisting}

\section{缩放与导航}
\begin{lstlisting}[style=lstCodeStylePlainText,title={缩放与导航}]
- 使用option + Cmd + 上和option + Cmd + 下进行放大和缩小,或者使用底部栏中的滑块。可以在触摸板上进行缩放
- 快速缩放:在非编辑note时,按下z键可缩小并查看整个文档。松开z,恢复之前缩放设置,此时聚焦松开z键时鼠标所悬停的文档部分
- 按住空格+鼠标左键可以滚动文档
\end{lstlisting}

\section{移动模式}
\begin{lstlisting}[style=lstCodeStylePlainText,title={移动模式}]
- 非编辑note状态,按下m,进入移动模式;(底部出现带四个箭头的十字图标,表明处于移动模式)
- 移动模式,按方向键会移动选中的笔记;
- 按住shift可使笔记移动更大距离
- 再次按下m可返回常规选择模式
\end{lstlisting}

%%%%%%%%%%%%%%%%%%%%%%%%%%%%%%%%%%%%%%%%%%%%%%%%%%%%%%%%%%%%%%%%%%%%%%%%%%%%%%%%%%%%%%%%%%%%%%%%%%%
\end{document}
