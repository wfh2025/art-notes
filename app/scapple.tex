\documentclass{../main.tex}{subfiles}
\begin{document}
%%%%%%%%%%%%%%%%%%%%%%%%%%%%%%%%%%%%%%%%%%%%%%%%%%%%%%%%%%%%%%%%%%%%%%%%%%%%%%%%%%%%%%%%%%%%%%%%%%%
\chapter{Scapple}
\section{概述}
\emph{Scapple}允许在虚拟纸张任意位置记录笔记,并能快速、轻松在这些笔记之间建立关联。
既能记录想法,又能梳理想法之间逻辑关系工具。

\section{基础}
\begin{enumerate}[itemsep=0pt, parsep=0pt, topsep=0pt, partopsep=0pt]
  \item 创建scapple文档
    \begin{itemize}[itemsep=0pt, parsep=0pt, topsep=0pt, partopsep=0pt]
      \item 菜单: File $\blacktriangleright$ New
      \item 快捷键: Command + N
    \end{itemize}
  \item 创建新note
    \begin{itemize}[itemsep=0pt, parsep=0pt, topsep=0pt, partopsep=0pt]
      \item 任意位置双击
      \item 快捷键: Command + Enter
    \end{itemize}
  \item 编辑note: 双击note
  \item 结束编辑note: 按下Escape
  \item 导入: Finder拖入图像文件/文本文件
  \item 选中note
    \begin{itemize}[itemsep=0pt, parsep=0pt, topsep=0pt, partopsep=0pt]
      \item 单个:点击
      \item 拖动选中多个: note外点击并拖动创建矩形,拖动矩形可选中多个note
      \item shift+点击或Command+点击,也能选中多个
    \end{itemize}
  \item 移动note
    \begin{itemize}[itemsep=0pt, parsep=0pt, topsep=0pt, partopsep=0pt]
      \item 选中note
      \item 拖动
    \end{itemize}
  \item 删除note
    \begin{itemize}[itemsep=0pt, parsep=0pt, topsep=0pt, partopsep=0pt]
      \item Delete
      \item 菜单: Edit $\blacktriangleright$ Delete
    \end{itemize}
  \item 连接操作
    \begin{itemize}[itemsep=0pt, parsep=0pt, topsep=0pt, partopsep=0pt]
      \item 将note拖到另外一个note,即可建立连接,重复操作会移除连接
      \item 将note拖到另外一个note,按住Option创建/删除箭头连接,
      \item 将note拖到另外一个note,拖放按住\texttt{Option+Command}可使箭头指向相反方向
      \item 将note拖到另外一个note,按住\texttt{Shift+Command}箭头双向指向
    \end{itemize}
  \item 创建新note使用修饰键,会使新note与选中note建立连接
    \begin{itemize}[itemsep=0pt, parsep=0pt, topsep=0pt, partopsep=0pt]
      \item 双击按住\texttt{Command},创建无箭头连接新note
      \item 抓住并拖动note之间的连接线,可移动相互连接的note
    \end{itemize}
  \item 双击两个note之间的连接线,可创建一个与两者都相连的新笔记,并替换原来的连接
  \item 点击并拖动note左边界或右边界可调整其大小;按住\texttt{Option}向两个方向调整大小
\end{enumerate}

\section{格式与排列}
\begin{lstlisting}[style=lstCodeStylePlainText]
- 使用检查器可以更改笔记以及整个文档的外观。
- 通过"视图">"显示检查器"来显示检查器.(检查器中许多选项也可以在"格式"菜单中找到);
- 可以通过"格式"菜单的"笔记样式"子菜单保存和应用格式预设。
- 可以通过"格式"菜单的"笔记样式"子菜单保存和应用格式预设。可以通过一个命令应用多种格式选项,笔记样式也可以通过上下文菜单应用。
- Notes菜单包含辅助排列note的命令,例如对齐、统一宽度/高度、置于顶层/底层
- Tab键缩进选中的note;Shift-Tab减少缩进
\end{lstlisting}

\section{堆叠}
\begin{lstlisting}[style=lstCodeStylePlainText]
- 选中一条笔记后,Cmd+Return可创建一条新笔记,直接堆叠其下方,当笔记处于堆叠状态,若上方笔记被编辑或移出堆叠,垂直位置会自动调整以保证堆叠状态
- 可以通过"Notes"菜单选择堆叠来堆叠选中的笔记(笔记会堆叠在最先选中的笔记下方)。
- 要将现有笔记添加到一个堆叠中,首先在你希望放置该笔记的堆叠下方选择一个笔记,然后选中要移动的笔记;从“Notes"菜单或上下文菜单中选择"堆叠".
\end{lstlisting}

\section{背景形状}
\begin{lstlisting}[style=lstCodeStylePlainText, title={背景形状}]
- 背景形状用于包围notes(或其他背景形状),作为视觉辅助。
- 要创建背景形状,按住Ctrl单击,新建>背景形状。(如果按住Ctrl单击现有notes或形状,该命令变为"围绕所选内容新建背景形状");
- 通过检查器或Notes菜单为背景形状开启“磁性”选项后,任何与该形状重叠的notes都会被吸附到它上面,因此移动该形状时嘛,重叠的note也会随之移动.
\end{lstlisting}

\section{缩放与导航}
\begin{lstlisting}[style=lstCodeStylePlainText,title={缩放与导航}]
- 使用option + Cmd + 上和option + Cmd + 下进行放大和缩小,或者使用底部栏中的滑块。可以在触摸板上进行缩放
- 快速缩放:在非编辑note时,按下z键可缩小并查看整个文档。松开z,恢复之前缩放设置,此时聚焦松开z键时鼠标所悬停的文档部分
- 按住空格+鼠标左键可以滚动文档
\end{lstlisting}

\section{移动模式}
\begin{lstlisting}[style=lstCodeStylePlainText,title={移动模式}]
- 非编辑note状态,按下m,进入移动模式;(底部出现带四个箭头的十字图标,表明处于移动模式)
- 移动模式,按方向键会移动选中的笔记;
- 按住shift可使笔记移动更大距离
- 再次按下m可返回常规选择模式
\end{lstlisting}

%%%%%%%%%%%%%%%%%%%%%%%%%%%%%%%%%%%%%%%%%%%%%%%%%%%%%%%%%%%%%%%%%%%%%%%%%%%%%%%%%%%%%%%%%%%%%%%%%%%
\end{document}
